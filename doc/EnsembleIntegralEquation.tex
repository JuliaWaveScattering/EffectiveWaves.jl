\pdfoutput=1

\documentclass[12pt, a4paper]{article}
\usepackage[utf8]{inputenc}

\usepackage[style= bwl-FU, backend = biber, maxbibnames=99, maxcitenames=2,uniquelist=minyear, firstinits=true]{biblatex}

% \AtEveryBibitem{\clearfield{number}}
% \AtEveryBibitem{\clearfield{doi}}
\AtEveryBibitem{\clearfield{note}}
\AtEveryBibitem{\clearfield{urldate}}
\AtEveryBibitem{\clearfield{issn}}
\AtEveryBibitem{\clearfield{isbn}}
%
\addbibresource{Library.bib}

\usepackage{authblk}
\usepackage{color,graphicx,tikz}
\usetikzlibrary{positioning,arrows}
% The amssymb package provides various useful mathematical symbols
\usepackage{mathtools,amssymb,amsmath,mathdots,amsfonts}
\usepackage[mathscr]{eucal} %just for the font \mathscr
\usepackage{enumerate}
%\usepackage{amsthm}
%\usepackage{mathaccents}
\usepackage{setspace}
\newtheorem{remark}{Remark} [section]
\usepackage{hyperref}

% macros used across many documents for multiple scattering notation
%% Packages for commenting and striking through text %%%%
\usepackage[colorinlistoftodos,bordercolor=orange,backgroundcolor=orange!20,linecolor=orange,textsize=scriptsize]{todonotes}
\usepackage{soul}
\newcommand{\will}[1]{\todo[inline]{\textbf{Will: }#1}}
\newcommand{\art}[1]{\todo[inline]{\textbf{Artur: }#1}}
\newcommand{\david}[1]{\todo[inline]{\textbf{David: }#1}}
\newcommand{\mike}[1]{\todo[inline]{\textbf{Mike: }#1}}
\newcommand{\wrong}[1]{\textcolor{red}{\st{#1}}}
%%%--------------------------------------------------%%%

\newcommand \eff {*}
\newcommand \scatZ {Z}
\newcommand \scatZs {\zeta}
\newcommand \Q {\vec Q}
\newcommand \T {\vec T}
\newcommand \B {\mathcal B}
\newcommand \regS {\mathcal S}
\newcommand \M {\vec {\mathcal M}}
\newcommand \s {\mathbf s}
\newcommand \Ab {\mathcal A}
\newcommand \A [1] {\Ab^#1}
\newcommand \p {p}
\newcommand \prob {P}
\newcommand {\type} [1]{{\{#1\}}}
\newcommand \reg {\mathcal R_N}
\newcommand \reginf {\mathcal R_\infty}
\newcommand {\nfrac}[1] {\mathfrak n_{#1}}
\newcommand {\Lamo}{\vec \Lambda}
\newcommand {\Lam}[1]{\Lamo_{#1}}
\newcommand \ef{\mathrm{E}}
\newcommand \reflect{\mathrm{ref}}
\newcommand \inc{\mathrm{in}}
\newcommand \In{\mathrm{I}}
\newcommand \cs { S}
\newcommand \cl { L}
\newcommand \Out{\mathrm{o}}

% \newcommand \nfrac {\mathfrak n_0}
\newcommand{\ensem}[1]{\langle #1 \rangle}

\def\bga#1\ega{\begin{gather}#1\end{gather}} % suggested in technote.tex
\def\bgas#1\egas{\begin{gather*}#1\end{gather*}}

\def\bal#1\eal{\begin{align}#1\end{align}} % suggested in technote.tex
\def\bals#1\eals{\begin{align*}#1\end{align*}}

\renewcommand{\vec}[1]{\boldsymbol{#1}}
\renewcommand{\thefootnote}{\fnsymbol{footnote}}

\newcommand{\initial}[1]{{#1}_\circ}
\newcommand{\ii}{\textrm{i}}
\newcommand{\ee}{\textrm{e}}
\newcommand{\deriv}[2]{\frac{\partial{#1}}{\partial{#2}}}
\newcommand{\derivtwo}[2]{\frac{\partial^{2}{#1}}{\partial{#2}^2}}
\newcommand{\derivtwomix}[3]{\frac{\partial^{2}{#1}}{\partial{#2}\partial{#3}}}

\newcommand{\bb}{\mathcal B}
\newcommand{\R}{\mathbb{R}}
\newcommand{\filler}{\hspace*{\fill}}

\newcommand{\lit}{\hspace{0.2cm}}

\newtheorem{theorem}{Theorem}
\def \proof{\noindent {\bf \emph{Proof:}} }


% \DeclareMathOperator{\sign}{sgn}
% \DeclareMathOperator{\divergence}{div}
% \DeclareMathOperator{\tr}{tr}
% \DeclareMathOperator{\Ord}{\mathcal{O}}
% \DeclareMathOperator{\GRAD}{grad}
% \DeclareMathOperator{\DIV}{DIV}
% \DeclarePairedDelimiter{\ceil}{\lceil}{\rceil}


%%% Local Variables:
%%% mode: latex
%%% TeX-master: t
%%% End:


% \graphicspath{{../images/}}
%\graphicspath{{Media/}}

\doublespacing
% \setlength{\topmargin}{0cm} \addtolength{\textheight}{2cm}
\evensidemargin=0cm \oddsidemargin=0cm \setlength{\textwidth}{16cm}

\begin{document}

\title{Numerically solving integral equations of wave ensembles}

% \author{
% Artur L. Gower$^{a}$, Michael J. A. Smith$^{a}$, \\ William Parnell$^{a}$ and David Abrahams$^{a,b}$ \\[12pt]
% \footnotesize{$^{a}$ School of Mathematics, University of Manchester, Oxford Road, Manchester, M13 9PL, UK}\\
% \footnotesize{$^{b}$ Isaac Newton Institute for Mathematical Sciences, 20 Clarkson Rd, Cambridge CB3 0EH, UK}
% }

\author[$\dagger$]{Artur L.\ Gower}

\affil[$\dagger$]{School of Mathematics, University of Manchester, Oxford Road, Manchester M13 9PL, UK}

\date{\today}
\maketitle

\begin{abstract}
Allows for a broad frequency range, and to easily test different statistical assumptions. Assumptions such as the pair-correlation and QCA.
\end{abstract}

\noindent
{\textit{Keywords:} polydisperse, multiple scattering, Fredholm Integral equations, multi-species, effective waves, quasicrystalline approximation, statistical methods}


\section{Effective waves for uniformly distributed species}
\label{sec:results}

We consider a halfspace $x>0$ filled with $S$ types of inclusions (species) that are uniformly distributed. The fields are governed by the scalar wave equation:
\begin{align}
  &\nabla^2 u + k^2 u = 0, \quad \text{(in the background material)} \\
  &\nabla^2 u + k^2_j u = 0, \quad \text{(inside the $j$-th scatterer)},
\end{align}
 The background and species material properties are summarised in Table~\ref{tab:properties}.
The goal is to calculate how a medium with these scatterers, randomly uniformly distributed, reflects and transmits waves in an
\href{https://en.wikipedia.org/wiki/Ensemble_average_(statistical_mechanics)}{ensemble average sense}.

For simplicity we will consider that all particles are cylindrical, though it is easy to extend the results to any smooth particle by using Waterman's T-matrix\cite{waterman_symmetry_1971,varadan_multiple_1978,mishchenko_t-matrix_1996}.

\begin{table}[h]
\centering
\begin{tabular}{|l| l|}
  \hline
Host material properties: &  wavenumber $k$ \hspace{0.8cm} density $\rho$  \hspace{0.25cm} sound speed $c$
\\ \hline
specie material properties: & number density $\nfrac j$ \hspace{0.1cm} density $\rho_j $
\hspace{0.1cm} sound speed $c_j$  \hspace{0.1cm} radius $a_j$
\\\hline
\multicolumn{2}{|l|}{
  total number density $\nfrac {}$   \hspace{0.2cm}  effective wavenumber $k_\eff$ \hspace{0.2cm} specie min. distance $a_{j\ell} > a_j + a_\ell$
}
\\\hline
% 2D Incident wave & $u_\inc(x,y) = \ee^{\ii \mathbf k \cdot \mathbf x} \quad \text{with} \quad \mathbf k \cdot \mathbf x = k x \cos \theta_\inc  + k y \sin \theta_\inc $
\end{tabular}
\label{tab:properties}
\caption{Summary of material properties and notation. The index $j$ refers to properties of the $j$-th specie. Note a typical choice for $a_{j\ell}$ is $a_{j\ell} = c (a_j + a_\ell)$, where $c=1.01$.}
\end{table}


\section{Cylindrical species}

We consider an incident wave
\begin{gather}
  u_\inc =  \ee^{\ii \mathbf k \cdot \mathbf x} \quad \text{with} \quad \mathbf k \cdot \mathbf x = k x \cos \theta_\inc  + k y \sin \theta_\inc,
%   u_\inc(x,y) = \ee^{\ii \mathbf k \cdot \mathbf x}, \quad \text{with} \quad \mathbf k \cdot \mathbf x = \alpha  x  + \beta y,
% \notag \\
%   \text{where} \qquad  \alpha = k \cos \theta_\inc, \quad \beta = k \sin \theta_\inc,
  \label{eqns:incident}
\end{gather}
and angle of incidence $\theta_\inc$ from the $x$-axis, exciting a material occupying the halfspace $x>0$.

Combining equations (3.6) and then quasicrystalline approximation (3.10) from \parencite{gower_reflection_2018}, we arrive at
\begin{multline}
\nfrac {} \sum_{n=-\infty}^\infty \int_\regS \int_{\stackrel{x_2 > 0 }{\|\mathbf x_1 - \mathbf x_2 \| > a_{21}}}  \A n (\mathbf x_2, \s_2) F_{n-m}(k\mathbf x_1 - k \mathbf x_2,k)  d \mathbf x_2 d\s_2^n
\\
+  \A m (\mathbf x_1, \s_1) + \ee^{\ii \mathbf x_1 \cdot \mathbf k } \ee^{\ii m ( \pi/2 - \theta_\inc )}
   = 0, \quad \text{for} \quad x_1 >0,
  \label{eqn:ensemAsystem}
\end{multline}
where
\begin{equation}
  F_{n-m}(k\mathbf x_1 - k \mathbf x_2,k) = \ee^{\ii (n-m) \Theta_{21}} H_{n-m}(k R_{21})g(\mathbf x_1 - \mathbf x_2| \s_1,\s_2),
  \label{eqn:integral_kernel}
\end{equation}
$p(\s_1)$ is the probability density function of picking a species in $\regS$ and we assumed statistical independence $p(\s_1, \s_2) = p(\s_1)p(\s_2)$. The function $g(\mathbf x_1 - \mathbf x_2| \s_1,\s_2)$ is the pair-correlation, assuming the particle centred at $\mathbf x_1$ ($\mathbf x_2$) is of type $\s_1$ ($\s_2$). If we were to use whole correction, then $g(\mathbf x_1 - \mathbf x_2| \s_1,\s_2) = 1$.

In terms of the notation from \parencite{gower_reflection_2018}:
\begin{multline}
  |\reg| p(\Lam 2 | \Lam 1) = |\reg|^2 \frac{p(\Lam 1, \Lam 2)}{p(\s_1)} = |\reg|^2 p(\s_2) p(\mathbf x_1, \mathbf x_2 | \s_1,\s_2) = p(\s_2) g(\mathbf x_1 - \mathbf x_2 | \s_1,\s_2).
\end{multline}

We also borrow equation (4.1) from \parencite{gower_reflection_2018} to substitute
\[
\A m (x_1, y_1, \s_1) = \A m (x_1, \s_1) \ee^{\ii y_1 k \sin\theta_\inc},
\]
which is a result of the symmetry present in~\eqref{eqn:ensemAsystem}. substituting the above into~\eqref{eqn:ensemAsystem} results in

\begin{multline}
  \int_{\stackrel{x_2 > 0 }{\|\mathbf x_2-\mathbf x_1 \| > a_{21}}}
  \A n (x_2, \s_2) \ee^{\ii y_2 k \sin\theta_\inc}
  F_{n-m}(k\mathbf x_1 - k \mathbf x_2,k) d \mathbf x_2 =
  \\
  \int_{x_2 > 0}
  \A n (x_2, \s_2)\int_{y \not \in B} \ee^{\ii y_2 k \sin\theta_\inc}
  F_{n-m}(k\mathbf x_1 - k \mathbf x_2,k) d y_2 d x_2,
\end{multline}
where $B$ is the interval
\[
B =
\begin{cases}
  [y_1 - \sqrt{a_{21}^2-(x_2-x_1)^2}, y_1 + \sqrt{a_{21}^2-(x_2-x_1)^2}], &  |x_2-x_1| \leq a_{21}  \\
  [0,0], &  |x_2-x_1| > a_{21}.
\end{cases}
\]


\begin{multline}
\nfrac {} \sum_{n=-\infty}^\infty \int_\regS \int_{\stackrel{x_2 > 0 }{\|\mathbf x_2-\mathbf x_1 \| > a_{21}}}
\A n (x_2, \s_2) \ee^{\ii y_2 k \sin\theta_\inc}
F_{n-m}(k\mathbf x_1 - k \mathbf x_2,k) d \mathbf x_2 d\s_2^n
\\
+  \A m (x_1, \s_1) \ee^{\ii y_1 k \sin\theta_\inc} + \ee^{\ii \mathbf x_1 \cdot \mathbf k } \ee^{\ii m ( \pi/2 - \theta_\inc )}
   = 0, \quad \text{for} \quad x_1 >0,
  \label{eqn:ensemAsystem2}
\end{multline}
% where $\mathbf X = \mathbf x_2 - \mathbf x_1$
\printbibliography

% \bibliographystyle{RS}
% \bibliography{../Library2}

\end{document}
